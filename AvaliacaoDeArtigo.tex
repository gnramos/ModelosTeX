%       @file: AvaliacaoDeArtigo.tex
%     @author: Guilherme N. Ramos (gnramos@unb.br)
%
% Modelo para avaliação de artigos da disciplina. Os trechos que devem ser
% alterados estão indicados com o comentário [<o>].
%
% Este documento usa a classe UnBExam, disponível em:
% https://github.com/gnramos/UnBExam
%
% Para um exemplo de fichamento completo, veja:
% https://github.com/UnB-CIC/Monografia/blob/master/doc/Fichamento.pdf


\documentclass{UnBExam}%

\documento{Avaliação de Artigo}%
\professor{Guilherme N. Ramos}%
\data{}%
\printanswers%

\begin{document}%
%%%%%%%%%%%%%%%%%%%%%%%%%%%%%%%%%%%%%%%%%%%%%%%%%%%%%%%%%%%%%%%%%%%%%%%%%%%%%
% Nem todo trabalho publicado é de qualidade ou mesmo tecnicamente correto. %
% Confie em seus conhecimentos e seja objetivo.                             %
%                                                                           %
% Para fazer a avaliação, não basta apenas ler o artigo, você deve conferir %
% os fatos mais importantes, se familiarizar com os assuntos para tentar    %
% conferir se os argumentos e conclusões apresentados são válidos. Para     %
% este trabalho, foque nos pontos mais importantes, os detalhes serão       %
% discutidos em sala de aula.                                               %
%%%%%%%%%%%%%%%%%%%%%%%%%%%%%%%%%%%%%%%%%%%%%%%%%%%%%%%%%%%%%%%%%%%%%%%%%%%%%

	\begin{description}%
		\item[Título:] % [<o>] Insira aqui o título do artigo sendo avaliado.

		\item[Autor(es):] % [<o>] Insira aqui a lista de autores do artigo sendo av
		                  %       aliado.

		\item[Revisor:] % [<o>] Insira aqui sua matrícula e nome.

		\item[Avaliação:] % [<o>] Escolha apenas uma das opções abaixo:
		Ruim | Médio | Bom | Excelente%
	\end{description}%

	\section{Interesse do Leitor}%
	\begin{enumerate}
		\item O artigo é interessante no contexto da disciplina?\\%
		% [<o>] Escolha apenas uma das opções abaixo:
		Sim | Talvez | Não

		\item Que tipo de mudança haverá neste interesse nos próximos 5 anos?\\%
		% [<o>] Escolha apenas uma das opções abaixo:
		Crescerá | Pouca mudança | Diminuirá

		\item Considerando o contexto específico do artigo, qual a importância
		do tópico abordado?\\%
		% [<o>] Escolha apenas uma das opções abaixo:
		Muita | Média | Pouca
	\end{enumerate}%

	\section{Conteúdo}%
	\begin{enumerate}
		\item O artigo é tecnicamente bem feito?\\%
		% [<o>] Escolha apenas uma das opções abaixo:
		Sim | Talvez | Parcialmente | Não

		\item Qual o nível de conhecimento técnico apresentado?\\%
		% [<o>] Escolha apenas uma das opções abaixo:
		Especialista | Avançado | Superficial

		\item O artigo apresenta uma contribuição tangível ao \emph{estado da
		arte} da área?\\%
		% [<o>] Escolha apenas uma das opções abaixo:
		Sim | Um pouco | Não

		\item As referências apresentadas são adequadas?\\%
		% [<o>] Escolha apenas uma das opções abaixo:
		Sim | Sim, mas veja a Seção~\ref{} | Não

		\item Qual a probabilidade do conteúdo apresentado ser utilizado por
		outros pesquisadores?\\%
		% [<o>] Escolha apenas uma das opções abaixo:
		Alta | Média | Baixa
	\end{enumerate}%

	% Considere esta seção apenas se houver análise experimental (ou simulações)
	\subsection{Experimentos}%
	\begin{enumerate}
		\item Os experimentos são claramente definidos?\\%
		% [<o>] Escolha apenas uma das opções abaixo:
		Sim | Não

		\item O procedimento adotado é descrito com detalhes suficientes para
		ser reproduzido?\\%
		% [<o>] Escolha apenas uma das opções abaixo:
		Sim | Não%
		\hfill\begin{minipage}{.29\textwidth}%

		% [<o>] Escolha apenas uma das opções abaixo:
		\hfill código disponível: Sim | Não%

		% [<o>] Escolha apenas uma das opções abaixo:
		\hfill dados disponíveis: Sim | Não%
		\end{minipage}%

		\item Os resultados são comparados ao \emph{estado da arte}?\\%
		% [<o>] Escolha apenas uma das opções abaixo:
		Sim | Não

		\item As métricas utilizadas são válidas?\\%
		% [<o>] Escolha apenas uma das opções abaixo:
		Sim | Não

		\item As conclusões são válidas?\\%
		% [<o>] Escolha apenas uma das opções abaixo:
		Sim | Não

		\item Os resultados são convincentes?\\%
		% [<o>] Escolha apenas uma das opções abaixo:
		Sim | Não
	\end{enumerate}%

	\subsection{Apresentação}%
	\begin{enumerate}
		\item O \emph{abstract} é um resumo suficiente e adequado do conteúdo
		apresentado?\\%
		% [<o>] Escolha apenas uma das opções abaixo:
		Sim | Não

		\item A introdução apresenta claramente o contexto e a motivação para
		alguém que não seja especialista?\\%
		% [<o>] Escolha apenas uma das opções abaixo:
		Sim | Não

		\item Como é a organização do artigo?\\%
		% [<o>] Escolha apenas uma das opções abaixo:
		Satisfatória | Pode melhorar | Ruim

		\item Considerando o conteúdo técnico, o tamanho do artigo é adequado?\\%
		% [<o>] Escolha apenas uma das opções abaixo:
		Sim | Muito longo | Muito curto

		\item A escrita é satisfatória?\\%
		% [<o>] Escolha apenas uma das opções abaixo:
		Sim | Não

		\item Qual a legibilidade do artigo pra um não especialista?\\%
		% [<o>] Escolha apenas uma das opções abaixo:
		Fácil | Autocontido | Precisa de esclarecimentos | Ilegível

		\item Desconsiderando o aspecto técnico, qual a qualidade do conteúdo
		apresentado?\\%
		% [<o>] Escolha apenas uma das opções abaixo:
		Excelente | Boa | Média | Ruim
	\end{enumerate}%

	\section{Resumo}%
	% [<o>] Escreva um resumo (1 a 2 páginas) do artigo.
	% A proposta é apresentar uma versão concisa dos conceitos mais relevantes
	% apresentados por seção, geralmente reduzindo cada uma a um parágrafo. O
	% resumo deve ser um texto fluido (e não uma lista de tópicos abordados).
	% Idealmente, o acesso a este resumo possibilita uma compreensão completa do
	% trabalho sem que seja necessário relê-lo.

	\section{Pontos-chave}%
	% Descreva os pontos chaves, conforme apresentados pelos autores no artigo.
	% Bons artigos deixam estes pontos bem claros ao longo do texto.

	\begin{description}
		\item[Proposta:]% [<o>] o que é proposto? (ex: algoritmo para
		                %       classificação supervisionada de dados
		                %       desbalanceados)

		\item[Mérito:]% [<o>] por que a proposta é relevante? Por que ela
		              %       deveria ser escolhida como solução de algum
		              %       problema ao invés de outra solução já existente?
		              %       (ex: usa abordagem XYZ na supervisão que a torna
		              %       mais acurada, aplicação em uma massa de dados
		              %       inédita)

		\item[Validação:]% [<o>] como verificar a utilidade da proposta (que ela
		                 %       realmente resolve o problema a que se aplica) e
		                 %       seu mérito (que realmente se destaca em algum
		                 %       aspecto)? (ex: validação cruzada e comparação
		                 %       com o estado da arte: algoritmo ABC)

		\item[Perspectivas:]% [<o>] o que pode ser melhorado? (ex: melhorias de
		                    %       eficiência e aplicação em domínio diferente)
	\end{description}%

	\section{Análise Crítica}%
	% [<o>] Escreva perguntas/ideias/comentários que você tenha sobre o artigo.
	% Por exemplo, trechos ou métodos que não são claros, pontos que deveriam
	% ser mais (ou menos) detalhados e tópicos que deveriam ter sido abordados
	% mas não foram (e como deveriam ter sido)

	% [<o>] Identifique pelo menos 3 pontos fortes e 3 pontos fracos do artigo.
	%
	% [<o>] Justifique sua avaliação inicial (Excelente | Bom | Médio | Ruim).
\end{document}%
