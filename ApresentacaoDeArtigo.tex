%       @file: ApresentacaoDeArtigo.tex
%     @author: Guilherme N. Ramos (gnramos@unb.br)
%
% Modelo para apresentação de artigos científicos. Exemplo de estrutura lógica.
%
% Este documento usa a classe UnBeamer, disponível em:
% https://github.com/UnB-CIC/UnBeamer
\documentclass{UnBeamer}%

%% ------------ Configurações ------------ %%
\usepackage[brazilian]{babel}%
\usepackage[utf8]{inputenc}%
\usepackage{lmodern}%

%% ------------ Autor, Título, etc. ------------ %%
\title[Resumo]{Apresentação de Artigo Científico}%
\author[gnramos]{Guilherme N. Ramos\\%
                 \texttt{gnramos@unb.br}}%
\institute[PPCA/CIC]{Departamento de Ciência da Computação\\%
                     Universidade de Brasília}%
\date[2018]{2018/01/01}%

%% ------------ Documento ------------ %%
\begin{document}%
    % Informações básicas do trabalho/apresentador(a).
    \frame{\titlepage}%

    % Resumo da apresentação.
    \begin{frame}%
        \frametitle{Resumo}%
        \tableofcontents%
    \end{frame}%

    % Resumo do trabalho.
    \begin{frame}%
        \frametitle{Pontos-chave}%
        % Descrição dos pontos chaves, conforme apresentados pelos autores no
        % artigo.

        \begin{description}
            \item[Proposta:] o que é proposto?\vfill% (ex: algoritmo para classificação supervisionada de dados desbalanceados)
            \item[Mérito:] por que a proposta é relevante? Por que ela deveria ser escolhida como solução de algum problema ao invés de outra solução já existente?\vfill% (ex: usa abordagem XYZ na supervisão que a torna mais acurada, aplicação em uma massa de dados inédita)
            \item[Validação:] como verificou-se a utilidade da proposta (que ela realmente resolve o problema a que se aplica) e seu mérito (que realmente se destaca em algum aspecto)?\vfill% (ex: validação cruzada e comparação com o estado da arte: algoritmo ABC))
            \item[Perspectivas:] o que pode ser melhorado?% (ex: melhorias de eficiência e aplicação em domínio diferente)
        \end{description}%
    \end{frame}%

    \section{Periódico}
    % Informações básicas e interessantes (se houver).
    \begin{frame}%
        Informações sobre o periódico.%
        \begin{itemize}%
            \item Editora%
            \item Edições (e periodicidade)%
            \item Fator de Impacto%
            \item Avaliação Qualis%
            \item Membros do corpo editorial de destaque%
            \item Outros trabalhos relevantes publicados%
            \item etc.%
        \end{itemize}%
    \end{frame}%

    \section{Artigo}%
    \begin{frame}%
        Detalhes do artigo:
        \begin{itemize}
            \item Dados da publicação
            \begin{itemize}%
                \item título%
                \item histórico de submissões%
                \item edição%
                \item páginas%
                \item DOI%
                \item etc.%
            \end{itemize}
            \item Número de citações
            \item etc.
        \end{itemize}
    \end{frame}%

    \subsection{Autores}%
    \begin{frame}%
        Informações sobre cada autor:
        \begin{itemize}
            \item Nome completo
            % \item \href{https://pt.wikipedia.org/wiki/%C3%8Dndice_h}{Índice H}%
            \item Página pessoal
            \item Instituição
            \item Trabalhos correlatos interessantes
            \item etc.
        \end{itemize}
    \end{frame}%

    \subsection{Abstract / Palavras chave}%
    \begin{frame}%
        Informações sobre o abstract e destaque das palavras-chave.
        % A leitura de um artigo é trabalhosa, a leitura do abstract e palavras-chave
        % salvam tempo e facilitam a pesquisa. A iedia aqui é, além de apresentar
        % os detalhes, indicar a sua motivação para ler o artigo a partir destas
        % informações.
    \end{frame}%

    % Seções do artigo.
    \subsection{Seção 1}%
    \begin{frame}%
        Informações apresentadas na primeira seção do artigo (geralmente \emph{Introdução}).%
        \\\vfill%
        Apresente o conteúdo, buscando esclarecer pontos mais complexos e complementar as informações quando necessário (por exemplo, indicando os trabalhos que serviram de referência ou acrescentando informações).
    \end{frame}%

    \subsection{Seção 2}%
    \begin{frame}%
        Informações apresentadas na seção (geralmente \emph{Trabalhos Correlatos/Estado da Arte}).%
        \\\vfill%
        Apresente o conteúdo, buscando esclarecer pontos mais complexos e complementar as informações quando necessário (por exemplo, indicando os trabalhos que serviram de referência ou acrescentando informações).
    \end{frame}%

    \subsection{Seção 3}%
    \begin{frame}%
        Informações apresentadas na seção (geralmente \emph{Proposta}).%
        \\\vfill%
        Apresente o conteúdo, buscando esclarecer pontos mais complexos e complementar as informações quando necessário (por exemplo, indicando os trabalhos que serviram de referência ou acrescentando informações).
    \end{frame}%

    \subsection{Seção 4}%
    \begin{frame}%
        Informações apresentadas na seção (geralmente \emph{Experimentos}).%
        \\\vfill%
        Apresente o conteúdo, buscando esclarecer pontos mais complexos e complementar as informações quando necessário (por exemplo, indicando os trabalhos que serviram de referência ou acrescentando informações).
    \end{frame}%

    \subsection{Seção 5}%
    \begin{frame}%
        Informações apresentadas na seção (geralmente \emph{Conclusões}).%
        \\\vfill%
        Apresente o conteúdo, buscando esclarecer pontos mais complexos e complementar as informações quando necessário (por exemplo, indicando os trabalhos que serviram de referência ou acrescentando informações).
    \end{frame}%

    % Cada seção deve ser apresentada e analisada.
    \subsection{$\cdots$}%

    \subsection{Referências}%
    \begin{frame}%
        Informações sobre as referências utilizadas:
        \begin{itemize}
            \item Quantidade
            \item Destaques
            \item etc.
        \end{itemize}
    \end{frame}%

    \section{Análise Crítica}%
    \begin{frame}%
        Explique o motivo de ter escolhido este artigo%
        \\\vfill%
        Forneça sua opinião pessoal sobre o artigo e uma análise crítica dele.
        \\\vfill%
        Indique os pontos fortes ou fracos que houver, sugerindo melhorias quando possível.
        \\\vfill%
        Por exemplo, trechos ou métodos que não são claros, pontos que deveriam
        ser mais (ou menos) detalhados e tópicos que deveriam ter sido abordados mas não foram (e como deveriam ter sido).
    \end{frame}%
\end{document}%
